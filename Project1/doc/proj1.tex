\documentclass[twocolumn]{article}[12pt]

\usepackage{amsmath}

\newcommand{\eq}[1]{\begin{align*}#1\end{align*}}
\renewcommand\vec[1]{{\bf #1}}
\newcommand{\OP}[1]{{\bf\widehat{#1}}}

\begin{document}

\section{Introduction}

\section{Mathematical Description}

%% Introduction to the problem
The Helium atom consists of two electrons orbiting a nucleus,
where we label the distance between electron 1 and the nucleus,
and electron 2 and the nucleus as
$r_1 = \sqrt{x_1^2 + y_1^2 + z_1^2}$ 
and $r_2 = \sqrt{x_2^2 + y_2^2 + z_2^2}$, respectively.

We model the total potential energy of the system as
\eq{
    V(r_1,r_2)=-\frac{2}{r_1}-\frac{2}{r_2}+\frac{1}{r_{12}}
}
Where the interaction between each electron and the nucleus
is given by the two first terms,
and the mutual electron-electron repulsion is given by the last.
The distance between the electrons is $r_{12}=|\vec r_1-\vec r_2|$.

The Hamiltonian of the system is then
\eq{
    \OP H = -\frac{\nabla_1 ^2}{2} -\frac{\nabla_2 ^2}{2}
    -\frac{2}{r_1}-\frac{2}{r_2}+\frac{1}{r_{12}}
}
%% Stating units / constants
The radii \{$r_1$, $r_2$, $r_{12}$\} 
of this project have been scaled, and are thus dimensionless.

%% The spherical nature of the wavefunction
The Laplace operator in three dimensions $\nabla^2$ can be represented as
\eq{
  \bigg( \frac{\partial^2}{\partial r^2} 
    + \frac{2}{r} \frac{\partial}{\partial r} \bigg) f
    +\frac{1}{r^2 \sin\theta}\frac{\partial}{\partial \theta}
    \bigg( \sin\theta \frac{\partial}{\partial \theta}  \bigg) f
    +\frac{1}{r^2 \sin^2\theta}\frac{\partial^2}{\partial^2 \phi} f
}
in sperical coordinates.

%% Calculating the integral using VMC and Metropolis algo.
\subsection{The Variational Principle}
The Variation Principle states that if we have a Hamiltonian
$\OP H$ and a trial wavefunction $\psi_{T}$,
an upper bound for the ground state energy is given by

\eq{E_0 \leq
   \langle H \rangle =
   \frac{\int d{\bf r_1}d{\bf r_2}\psi^{\ast}_{Ti}({\bf r_1},{\bf r_2}, 
       {\bf r_{12}})
   \OP{H}({\bf r_1},{\bf r_2}, {\bf r_{12}})
   \psi_{Ti}({\bf r_1},{\bf r_2}, {\bf r_{12}})}
   {\int d{\bf r_1}d{\bf r_2}\psi^{\ast}_{Ti}({\bf r_1},{\bf r_2}, {\bf r_{12}})
   \psi_{Ti}({\bf r_1},{\bf r_2}, {\bf r_{12}})}.
}

\subsection{Variational Monte Carlo (VMC)}
The integrals to be solved in the variational method,
are often too large to be solved by traditional integral methods.

Therefore, we introduce the brute-force Monte Carlo method
to solve the integrals.

The 


%% End of the VMC section

To conclude if the computational methods are implemented correctly,
we should check that the results are reasonable. We do this by
finding a mathematical approximation of the closed form expression.

%% The closed-form expression.
Given the trial wavefunction $\psi_T (\vec R, \boldsymbol\alpha)$, 
we define a new quantity
\eq{
  E_L = \frac{1}{\psi_T}\OP H \psi_T
}
where $\boldsymbol\alpha$ is a set of variational parameters.
$E_{L1}$ is called the local energy

\subsection{The first trial wavefunction}
We first model the variational solution with a trial function of one
variation parameter $\alpha$. It has the form
\eq{
\psi_{T1}({\bf r_1},{\bf r_2}) = 
   \exp{\left(-\alpha(r_1+r_2)\right)}
}

The only part of the operator $\OP H$ that affects the wavefunction
are the Laplace operators.

Since $\psi_{T1}$ is only spatially dependent on $r_1$ and $r_2$,
the Laplaces of $\psi_{T1}$ reduces to
\eq{
  \nabla_i^2 \psi_{T1} = \bigg( \frac{\partial^2}{\partial r_i^2} 
    + \frac{2}{r_i} \frac{\partial}{\partial r_i} \bigg) \psi_{T1}
    = \bigg( \alpha^2 -\alpha\frac{2}{r_i}  \bigg)\psi_{T1}
}
for $i = 1,\;2$, since
\eq{
  \frac{\partial}{\partial r_i} e^{-\alpha (r_1+r_2)}
    &= -\alpha e^{-\alpha (r_1+r_2)}\\
\frac{\partial^2}{\partial r_i^2} e^{-\alpha (r_1+r_2)}
    &= \alpha^2 e^{-\alpha (r_1+r_2)}
}
This gives us the following trial energy
\eq{
  E_{L1}&=\frac{1}{\psi_{T1}}\bigg( -\alpha^2 
  +\alpha\bigg( \frac{1}{r_1}+\frac{1}{r_1}  \bigg)
    -\frac{2}{r_1}-\frac{2}{r_2} + \frac{1}{r_{12}}
    \bigg)\psi_{T1}\\
  &=(\alpha-2)\bigg( \frac{1}{r_1}+\frac{1}{r_2} \bigg)
    +\frac{1}{r_{12}}-\alpha^2
}
The $2$ in the $\alpha-2$ term is the number of protons, Z.

\subsection{The second trial wavefunction}
To approximate the closed-form solution even better,
we assume another trial wavefunction based on $\psi_{T1}$, namely
\eq{
  \psi_{T2} ({\bf r_1},{\bf r_2}, {\bf r_{12}})
    =\exp{\left(-\alpha(r_1+r_2)\right)}
    \exp{\left(\frac{r_{12}}{2(1+\beta r_{12})}\right)}
}
The second part is dependent on the distance between the
electrons.

\section{}
\end{document}
