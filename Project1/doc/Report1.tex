\documentclass[twocolumn]{article}[10pt]
\usepackage[english]{babel} 
\usepackage[latin1]{inputenc} 
\usepackage{times} 			% Default times font style
\usepackage[T1]{fontenc} 	% Font encoding
\usepackage{amsmath} 		% Math package
\usepackage{mathtools} 		% Adds the declare paired 
							% delimeter command to make costom \abs and \norm
\usepackage{breqn}		 	% Adds dmath environment for automated brakeline
\usepackage{xfrac}			% Adds slanted fractions (sfrac)
\usepackage{cancel}			% Adds the cancel command, a slash through the symbol(s)
\usepackage{tabularx}		% Adds adjustable width on tabulars
\usepackage{cuted}			% Adds the strip command, pagewidth text in a twocolumn
							% environment. 

% Start costum \abs \norm 
\DeclarePairedDelimiter\abs{\lvert}{\rvert}%
\DeclarePairedDelimiter\norm{\lVert}{\rVert}%
% Swap the definition of \abs* and \norm*, so that \abs
% and \norm resizes the size of the brackets, and the 
% starred version does not.
\makeatletter
\let\oldabs\abs
\def\abs{\@ifstar{\oldabs}{\oldabs*}}
%
\let\oldnorm\norm
\def\norm{\@ifstar{\oldnorm}{\oldnorm*}}
\makeatother
% End costum \abs \norm 

% TODO: Put comments on this section.
\newcommand{\eq}[1]{\begin{align*}#1\end{align*}}
\newcommand{\equ}[1]{\begin{align}#1\end{align}}
\renewcommand\vec[1]{{\bf #1}}
\newcommand{\OP}[1]{{\bf\widehat{#1}}}

\begin{document}
\begin{abstract}
Write something interesting about the main findings of the project.
\end{abstract}


\section{Introduction}
This is part 1 of a 3 part project to evaluate the ground state properties of 
some single atoms and diatomic molecules, 
using variational Monte Carlo simulations. 

A compulsory part of doing this is using expandable programs that can handle 
systems of increasing particles and complexity. 
This poses many challenges; implementational, statistical, 
numerical and mathematical. 
The atoms studied here are Helium and Beryllium, 
and the ground state energies and one-body densities are the properties
we are interested in. The computational methods to do this is a main focus of 
part 1. 

We are interested in the ground state energies, because the
results are more easily verifiable. One aim of the project is
to see how effective our numerical methods are in computing
these quantities, and compare the methods to each other,
by applying statistical tools.

As stated, there are several challenges. Here basic mathematical,
computational, statistical models are dealt with, and applied to the second 
most simple system, Helium. 

All results displayed in this article are 
gathered from our variational Monte Carlo program. 

\section{Methods}
%% Introduction to the problem
The Helium atom consists of two electrons orbiting a nucleus,
where the distance between electron 1 and the nucleus,
and electron 2 and the nucleus are labeled as
$r_1 = \sqrt{x_1^2 + y_1^2 + z_1^2}$ 
and $r_2 = \sqrt{x_2^2 + y_2^2 + z_2^2}$.

The total potential energy of the system is modelled as
{\small
\eq{
    V(r_1,r_2)=-\frac{2}{r_1}-\frac{2}{r_2}+\frac{1}{r_{12}}
}}%
where the interaction between each electron and the nucleus
is given by the two first terms. 
The mutual electron-electron repulsion is given by the last.
The distance between the electrons is $r_{12}=|\vec r_1-\vec r_2|$.

The Hamiltonian of the system is thusly, 
{\small
\eq{
    \OP H = -\frac{\nabla_1 ^2}{2} -\frac{\nabla_2 ^2}{2}
    -\frac{2}{r_1}-\frac{2}{r_2}+\frac{1}{r_{12}}
}}%

%% Stating units / constants
The radii \{$r_1$, $r_2$, $r_{12}$\} are scaled, and are dimensionless.

%% The spherical nature of the wavefunction
The Laplace operator of a function $f$ in three dimensions, $\nabla^2 f$,
can be represented as
{\small
\eq{
  \bigg( \frac{\partial^2}{\partial r^2} 
    + \frac{2}{r} \frac{\partial}{\partial r} \bigg) f
    +\frac{1}{r^2 \sin\theta}\frac{\partial}{\partial \theta}
    \bigg( \sin\theta \frac{\partial}{\partial \theta}  \bigg) f
    +\frac{1}{r^2 \sin^2\theta}\frac{\partial^2}{\partial^2 \varphi} f
}}%

in spherical coordinates, and as

{\small
\eq{
	\bigg( 
	\frac{\partial}{\partial x^2} +
	\frac{\partial}{\partial y^2} +
	\frac{\partial}{\partial z^2}
	\bigg) f
}}%

in cartesian coordinates.

%% Calculating the integral using VMC and Metropolis algo.
\subsection{The Variational Principle}
The Variation Principle states that if we have a Hamiltonian
$\OP H$ and a trial wavefunction $\psi_{T}$,
an upper bound for the ground state energy is given by
equation (\ref{eq:E0var})
\begin{strip}
{
\equ{
	&E_0 \leq
	\langle H \rangle =
	\frac{\int d{\bf r_1}d{\bf r_2}\psi^{\ast}_{Ti}({\bf r_1},{\bf r_2}, 
	{\bf r_{12}})
	\OP{H}({\bf r_1},{\bf r_2}, {\bf r_{12}})
	\psi_{Ti}({\bf r_1},{\bf r_2}, {\bf r_{12}})}
	{\int d{\bf r_1}d{\bf r_2}\psi^{\ast}_{Ti}({\bf r_1},{\bf r_2}, {\bf r_{12}})
	\psi_{Ti}({\bf r_1},{\bf r_2}, {\bf r_{12}})}\label{eq:E0var}
}}%
\end{strip}


\subsection{Variational Monte Carlo (VMC)}
The integrals to be solved in the variational method,
are often too large to be solved by traditional integral methods.

Therefore, we introduce the brute-force Monte Carlo method
to solve the integrals.

\subsubsection{VMC Algorithm}

\begin{itemize}
    \item Initialization.
    \begin{itemize}    
        \item Set a fixed number of MC steps.
        \item Choose initial position $\vec R$ and variational
            parameters $\boldsymbol\alpha$.
        \item Calculate $|\psi_T(\vec R)|^2$
    \end{itemize}
    \item Set energy and variance and start the MC calculation.
    \begin{itemize}    
        \item Find the trial position $\vec R' =\vec R + \delta\times r $,
            where $r\in [0,1]$ is randomly selected.
        \item Use the Metropolis algorithm to determine if the
            move $w =\frac{P(\vec R')}{P(\vec R)}$ is accepted or rejected.
        \item Given that the move is accepted, set $\vec R = \vec R'$.
        \item Update averages.
    \end{itemize}
    \item Compute final averages.
\end{itemize}
%% End of the VMC section

\subsection{The Metropolis Algorithm}

\subsection{Importance Sampling}

\subsection{Statistical Analysis}
The MC calculation are a set of computational \textit{experiments}, 
and to evaluate the results 
we need to implement statistical analysis of these experiments
to find \textit{statistical} errors.

For this purpose, we implement the \textit{blocking technique}.


\subsubsection{Blocking Implementation}

\begin{itemize}
    \item Compute MC calculation, store samples in array.
    \item Loop over a set of block sizes $n_b$.
    \item For each $n_b$, calculate the mean of the block
        and store these values in a new array.
    \item Take the mean and variance of this array.
    \item Store results.
\end{itemize}

\subsection{The Trial Wavefunctions for Helium}

To conclude if the computational methods are implemented correctly,
we should check that the results are reasonable. We do this by
finding a mathematical approximation of the closed form expression.

%% The closed-form expression.
Given the trial wavefunction $\psi_T (\vec R, \boldsymbol\alpha)$, 
we define a new quantity
{\small
\eq{
  E_L = \frac{1}{\psi_T}\OP H \psi_T
}}%
where $\boldsymbol\alpha$ is a set of variational parameters.
$E_{L1}$ is called the local energy

\subsubsection{The first trial wavefunction}
We first model the variational solution with a trial function of one
variation parameter $\alpha$. It has the form
{\small
\eq{
\psi_{T1}({\bf r_1},{\bf r_2}) = 
   \exp{\left(-\alpha(r_1+r_2)\right)}
}}%

The only part of the operator $\OP H$ that affects the wavefunction
are the Laplace operators.

Since $\psi_{T1}$ is only spatially dependent on $r_1$ and $r_2$,
the Laplaces of $\psi_{T1}$ reduces to
{\small
\eq{
  \nabla_i^2 \psi_{T1} = \bigg( \frac{\partial^2}{\partial r_i^2} 
    + \frac{2}{r_i} \frac{\partial}{\partial r_i} \bigg) \psi_{T1}
    = \bigg( \alpha^2 -\alpha\frac{2}{r_i}  \bigg)\psi_{T1}
}}%
for $i = 1,\;2$, since
{\small
\eq{
  \frac{\partial}{\partial r_i} e^{-\alpha (r_1+r_2)}
    &= -\alpha e^{-\alpha (r_1+r_2)}\\
\frac{\partial^2}{\partial r_i^2} e^{-\alpha (r_1+r_2)}
    &= \alpha^2 e^{-\alpha (r_1+r_2)}
}}%
This gives us the following trial energy
{\small
\eq{
  E_{L1}&=\frac{1}{\psi_{T1}}\bigg( -\alpha^2 
  +\alpha\bigg( \frac{1}{r_1}+\frac{1}{r_1}  \bigg)
    -\frac{2}{r_1}-\frac{2}{r_2} + \frac{1}{r_{12}}
    \bigg)\psi_{T1}\\
  &=(\alpha-2)\bigg( \frac{1}{r_1}+\frac{1}{r_2} \bigg)
    +\frac{1}{r_{12}}-\alpha^2
}}%
The $2$ in the $\alpha-2$ term is the number of protons, Z.

\subsubsection{The second trial wavefunction}
To approximate the closed-form solution even better,
we assume another trial wavefunction based on $\psi_{T1}$, namely
{\small
\eq{
  \psi_{T2} ({\bf r_1},{\bf r_2}, {\bf r_{12}})
    =\exp{\left(-\alpha(r_1+r_2)\right)}
    \exp{\left(\frac{r_{12}}{2(1+\beta r_{12})}\right)}
}}%
The second part is dependent on the distance between the
electrons, and is called the correlation part,
which accounts for the effect between the electrons.

One can then in the same way as for $\psi_{T1}$ calculate
the local energy. The correlations part will give us some trouble
when we try to calculate the Laplacian. This is due to
the distance between $\vec r_1$ and $\vec r_2$, since this quantity
is dependent on the angles $\varphi$ and $\theta$.
It has the form
{\small
\eq{
	E_{L2} = E_{L1}+\frac{1}{2(1+\beta r_{12})^2}
	\bigg(\frac{\alpha(r_1+r_2)}{r_{12}}(1-
	\frac{\mathbf{r}_1^T\mathbf{r}_2}{r_1r_2})\\
	-\frac{1}{2(1+\beta r_{12})^2}-\frac{2}{r_{12}}+
	\frac{2\beta}{1+\beta r_{12}}\bigg)
}}%

\subsection{The Trial Wavefunction for Beryllium}
We can use a trial wavefunction on the same form as for
Hydrogen and Helium

{\small
\eq{
	\psi_{T}({\bf r_1},{\bf r_2}, {\bf r_3}, {\bf r_4}) &= 
 	Det\left(\phi_{1}({\bf r_1}),\phi_{2}({\bf r_2}),
	\phi_{3}({\bf r_3}),\phi_{4}({\bf r_4})\right)\\ &\cdot
   	\prod_{i<j}^{4}\exp{\left(\frac{r_{ij}}{2(1+\beta r_{ij})}\right)}
}}%

where we approximate the Slater determinant \textit{Det} as
{\small
\eq{
	\psi_{T}({\bf r_1},{\bf r_2}, {\bf r_3}, {\bf r_4})&\propto 
	\left(\phi_{1s}({\bf r_1})\phi_{2s}({\bf r_2})-\phi_{1s}({\bf r_2})
	\phi_{2s}({\bf r_1})\right)\\
	&\cdot\left(\phi_{1s}({\bf r_3})\phi_{2s}({\bf r_4})
	-\phi_{1s}({\bf r_4})\phi_{2s}({\bf r_3})\right)
}}%
where the hydrogenic wavefunctions for the two spin orbitals
are given by
{\small\eq{
\phi_{1s}({\bf r_i}) = e^{-\alpha r_i}
}} and
{\small\eq{
\phi_{2s}({\bf r_i}) = \left(1-\alpha r_i/2\right)e^{-\alpha r_i/2}
}}

\subsection{Onebody Density and Charge Density}
The one-body density is computed from the form 
{\small
\eq{
	\rho(\vec R) = |\psi(\vec R)|^2
}}%

This is implemented as NOTE: ... in our program. 

\subsection{Implementation}
The methods described above are implemented in an object oriented C++ program
which is simple to use, and does not have any dependencies. 

Most of the code is contained in the class VMCSolver.cpp and VMCSolver.h. 
These were created to contain the full system and parameters for a single run. 

To start a simulation one must instantiate the solver and 
initialize the system, either by file and/or manually. After this
the integration can be run and all data will be contained in the solver object.
The last step is to collect data from the solver. 

All plot data are generated by individual programs using the mentioned solver. 
Plots are generated using python. 

\section{Results}
\subsection{Helium: Estimating $\alpha$ and $\beta$}
These parameters decide the minimal ground energy for the system, and both
should be adjusted simultaneously to reach the most optimal value of the
energy. First we sample the parameterspace of the first trial wave function. 
This function had a minimum of about NOTE: () + statistical inaccuracy here???
This was the first value of $\alpha$ that was used in sampling the 
parameterspace with beta in the second trial wavefunction. 
Using the computed $\beta$-value we again
sampled the energy as a function of $\alpha$. NOTE: This can be seen in ...

FOOTNOTE\footnote{Finding the statistically most optimal values of $\beta$ 
is challenging, because the mean variance is
substantial compared to the energies around the flat region.}   



\subsection{Helium: Comparing $\psi_{T1}$ and $\psi_{T2}$ }
Firstly the first and second trial wavefunctions for Helium are compared. 
The difference between these two are that there are no correlation term
in the first wavefunction. This makes computations much faster, but we
would expect that our results are farther away from the experimental 
value of the ground state. This is done using a numerical approximation 
of the local energy of the particles. 

\subsection{Helium: Closed Form Local Energy}
The local energy has a closed form solution for Helium, this has 
been implemented in fig.... Here we can see something. 

\subsection{Helium: Importance Sampling}
Now we would like to see how the program does with importance sampling instead 
of static steplength. This we can see in NOTE: fig ... 

\subsection{Helium: Evaluating Ground State and Density}
With our optimal parameters, we see that using NOTE: something something gives the
lowest variance of the mean and the lowest mean energy. 
Compared to experimental values, this result is statistically significant. 


\section{Discussion}
\subsection{Parameter Space}
After estimating the parameters in the start of Helium, 
we didn't adjust them anymore.
The standard deviation of the mean was not considered when estimating the values 
of alpha and beta. Performing blocking with our program would require full 
sampling of every step which stores about 10 mb for every value of alpha and 
beta. 
\subsection{Testing of the Solver}
Here we did unittests. 


\end{document}
