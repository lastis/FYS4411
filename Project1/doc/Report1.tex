\documentclass[twocolumn]{article}[12pt]

\usepackage{amsmath}

\newcommand{\eq}[1]{\begin{align*}#1\end{align*}}
\renewcommand\vec[1]{{\bf #1}}
\newcommand{\OP}[1]{{\bf\widehat{#1}}}

\begin{document}

\section{Introduction}

\section{Methods}

%% Introduction to the problem
The Helium atom consists of two electrons orbiting a nucleus,
where we label the distance between electron 1 and the nucleus,
and electron 2 and the nucleus as
$r_1 = \sqrt{x_1^2 + y_1^2 + z_1^2}$ 
and $r_2 = \sqrt{x_2^2 + y_2^2 + z_2^2}$, respectively.

We model the total potential energy of the system as
{\small
\eq{
    V(r_1,r_2)=-\frac{2}{r_1}-\frac{2}{r_2}+\frac{1}{r_{12}}
}}%
Where the interaction between each electron and the nucleus
is given by the two first terms,
and the mutual electron-electron repulsion is given by the last.
The distance between the electrons is $r_{12}=|\vec r_1-\vec r_2|$.

The Hamiltonian of the system is then
{\small
\eq{
    \OP H = -\frac{\nabla_1 ^2}{2} -\frac{\nabla_2 ^2}{2}
    -\frac{2}{r_1}-\frac{2}{r_2}+\frac{1}{r_{12}}
}}%

%% Stating units / constants
The radii \{$r_1$, $r_2$, $r_{12}$\} 
of this project have been scaled, and are thus dimensionless.

%% The spherical nature of the wavefunction
The Laplace operator of a function $f$ in three dimensions, $\nabla^2 f$,
can be represented as
{\small
\eq{
  \bigg( \frac{\partial^2}{\partial r^2} 
    + \frac{2}{r} \frac{\partial}{\partial r} \bigg) f
    +\frac{1}{r^2 \sin\theta}\frac{\partial}{\partial \theta}
    \bigg( \sin\theta \frac{\partial}{\partial \theta}  \bigg) f
    +\frac{1}{r^2 \sin^2\theta}\frac{\partial^2}{\partial^2 \varphi} f
}}%
in sperical coordinates, and as
{\small
\eq{
	\bigg( 
	\frac{\partial}{\partial x^2} +
	\frac{\partial}{\partial y^2} +
	\frac{\partial}{\partial z^2}
	\bigg) f
}}%
in cartesian coordinates.
i
%% Calculating the integral using VMC and Metropolis algo.
\subsection{The Variational Principle}
The Variation Principle states that if we have a Hamiltonian
$\OP H$ and a trial wavefunction $\psi_{T}$,
an upper bound for the ground state energy is given by
{\small
\eq{
	&E_0 \leq
	\langle H \rangle \\&=
	\frac{\int d{\bf r_1}d{\bf r_2}\psi^{\ast}_{Ti}({\bf r_1},{\bf r_2}, 
	{\bf r_{12}})
	\OP{H}({\bf r_1},{\bf r_2}, {\bf r_{12}})
	\psi_{Ti}({\bf r_1},{\bf r_2}, {\bf r_{12}})}
	{\int d{\bf r_1}d{\bf r_2}\psi^{\ast}_{Ti}({\bf r_1},{\bf r_2}, {\bf r_{12}})
	\psi_{Ti}({\bf r_1},{\bf r_2}, {\bf r_{12}})}
}}%

\subsection{Variational Monte Carlo (VMC)}
The integrals to be solved in the variational method,
are often too large to be solved by traditional integral methods.

Therefore, we introduce the brute-force Monte Carlo method
to solve the integrals.

\subsubsection{VMC Algorithm}

\begin{itemize}
    \item Initialization.
    \begin{itemize}    
        \item Set a fixed number of MC steps.
        \item Choose initial position $\vec R$ and variational
            parameters $\boldsymbol\alpha$.
        \item Calculate $|\psi_T(\vec R)|^2$
    \end{itemize}
    \item Set energy and variance and start the MC calculation.
    \begin{itemize}    
        \item Find the trial position $\vec R' =\vec R + \delta\times r $,
            where $r\in [0,1]$ is randomly selected.
        \item Use the Metropolis algorithm to determine if the
            move $w =\frac{P(\vec R')}{P(\vec R)}$ is accepted or rejected.
        \item Given that the move is accepted, set $\vec R = \vec R'$.
        \item Update averages.
    \end{itemize}
    \item Compute final averages.
\end{itemize}
%% End of the VMC section

\subsection{The Metropolis Algorithm}

\subsection{Importance Sampling}

\subsection{Statistical Analysis}
The MC calculation are a set of computational \textit{experiments}, 
and to evaluate the results 
we need to implement statistical analysis of these experiments
to find \textit{statistical} errors.

For this purpose, we implement the \textit{blocking technique}.


\subsubsection{Blocking Implementation}

\begin{itemize}
    \item Compute MC calculation, store samples in array.
    \item Loop over a set of block sizes $n_b$.
    \item For each $n_b$, calculate the mean of the block
        and store these values in a new array.
    \item Take the mean and variance of this array.
    \item Store results.
\end{itemize}

\subsection{The Trial Wavefunctions}

To conclude if the computational methods are implemented correctly,
we should check that the results are reasonable. We do this by
finding a mathematical approximation of the closed form expression.

%% The closed-form expression.
Given the trial wavefunction $\psi_T (\vec R, \boldsymbol\alpha)$, 
we define a new quantity
{\small
\eq{
  E_L = \frac{1}{\psi_T}\OP H \psi_T
}}%
where $\boldsymbol\alpha$ is a set of variational parameters.
$E_{L1}$ is called the local energy

\subsubsection{The first trial wavefunction}
We first model the variational solution with a trial function of one
variation parameter $\alpha$. It has the form
{\small
\eq{
\psi_{T1}({\bf r_1},{\bf r_2}) = 
   \exp{\left(-\alpha(r_1+r_2)\right)}
}}%

The only part of the operator $\OP H$ that affects the wavefunction
are the Laplace operators.

Since $\psi_{T1}$ is only spatially dependent on $r_1$ and $r_2$,
the Laplaces of $\psi_{T1}$ reduces to
{\small
\eq{
  \nabla_i^2 \psi_{T1} = \bigg( \frac{\partial^2}{\partial r_i^2} 
    + \frac{2}{r_i} \frac{\partial}{\partial r_i} \bigg) \psi_{T1}
    = \bigg( \alpha^2 -\alpha\frac{2}{r_i}  \bigg)\psi_{T1}
}}%
for $i = 1,\;2$, since
{\small
\eq{
  \frac{\partial}{\partial r_i} e^{-\alpha (r_1+r_2)}
    &= -\alpha e^{-\alpha (r_1+r_2)}\\
\frac{\partial^2}{\partial r_i^2} e^{-\alpha (r_1+r_2)}
    &= \alpha^2 e^{-\alpha (r_1+r_2)}
}}%
This gives us the following trial energy
{\small
\eq{
  E_{L1}&=\frac{1}{\psi_{T1}}\bigg( -\alpha^2 
  +\alpha\bigg( \frac{1}{r_1}+\frac{1}{r_1}  \bigg)
    -\frac{2}{r_1}-\frac{2}{r_2} + \frac{1}{r_{12}}
    \bigg)\psi_{T1}\\
  &=(\alpha-2)\bigg( \frac{1}{r_1}+\frac{1}{r_2} \bigg)
    +\frac{1}{r_{12}}-\alpha^2
}}%
The $2$ in the $\alpha-2$ term is the number of protons, Z.

\subsubsection{The second trial wavefunction}
To approximate the closed-form solution even better,
we assume another trial wavefunction based on $\psi_{T1}$, namely
{\small
\eq{
  \psi_{T2} ({\bf r_1},{\bf r_2}, {\bf r_{12}})
    =\exp{\left(-\alpha(r_1+r_2)\right)}
    \exp{\left(\frac{r_{12}}{2(1+\beta r_{12})}\right)}
}}%
The second part is dependent on the distance between the
electrons, and is called the correlation part,
which accounts for the effect between the electrons.

One can then in the same way as for $\psi_{T1}$ calculate
the local energy. The correlations part will give us some trouble
when we try to calculate the Laplacian. This is due to
the distance between $\vec r_1$ and $\vec r_2$, since this quantity
is dependent on the angles $\varphi$ and $\theta$.
It has the form
{\small
\eq{
	E_{L2} = E_{L1}+\frac{1}{2(1+\beta r_{12})^2}
	\bigg(\frac{\alpha(r_1+r_2)}{r_{12}}(1-
	\frac{\mathbf{r}_1^T\mathbf{r}_2}{r_1r_2})\\
	-\frac{1}{2(1+\beta r_{12})^2}-\frac{2}{r_{12}}+
	\frac{2\beta}{1+\beta r_{12}}\bigg)
}}%

\section{Results}

\section{Discussion}

\end{document}
